\documentclass[11pt]{article}
\usepackage[utf8]{inputenc}
\usepackage[T1]{fontenc}
\usepackage[frenchb]{babel}                   
\usepackage{hyperref}
\usepackage{geometry}
\usepackage{xcolor}
\usepackage{newpxtext,newpxmath}
\hypersetup{
  	pdfauthor = {Louis Viot},
	pdftitle = {Louis Viot : Lettre de motivation},
  	pdfsubject = {Lettre de motivation},
         colorlinks=true,
	allcolors=green!50!black,     
        urlcolor=red!50!black           
}
\geometry{
	left=2.0cm,
	right=2.0cm,
	top=1.0cm
}
\setlength{\parindent}{0cm}
\setlength{\parskip}{0.2cm}
\pagestyle{empty}

\begin{document}
\begin{flushleft}
	\textbf{Louis Viot}\\
	30 Avenue Henri Pontier \\
	Batîment C2 \\
	Aix-en-Provence, 13100 France\\
	Téléphone: (+33)628685904 \\
	E-mail: \href{mailto:louis.viot@alumni.enseeiht.fr}{louis.viot@alumni.enseeiht.fr} \\
\end{flushleft}
\begin{flushright}
	CEA Cadarache,\\
	13115 Saint-Paul-lez-Durance\\
\end{flushright}
\begin{flushright}
À Aix-en-Provence, le \today{}.
\end{flushright}
\textbf{Objet :} Candidature au poste ``Ingénieur développement et validation de code de calcul accidents graves''

Madame, Monsieur,

Souhaitant poursuivre ma carrière de chercheur au CEA, je souhaite vous proposer ma candidature au poste intitulé ``Ingénieur développement et validation de code de calcul accidents graves'' à pourvoir au sein du LMAG, laboratoire dans lequel je finis actuellement ma thèse de doctorat.

Ayant obtenu un diplôme d'ingénieur à l'ENSEEIHT en spécialité informatique et mathématiques appliquées, j'ai suivi de nombreux cours de développement logiciel et d'analyse numérique. Je possède donc cette double formation me permettant d'une part de résoudre des problèmes numériques liés à la modélisation physique et d'autre part de participer activement au développement d'un code de calcul. Durant les trois ans de thèse, j'ai d'une part créé et intégré une nouvelle architecture logicielle au sein de la plateforme PROCOR permettant le couplage en temps des différents modèles physiques intervenant lors d'un accident grave et d'autre part étudié les schémas numériques permettant la résolution de ce couplage. J'ai donc dû manipuler l'ensemble de ces modèles, par exemple les modèles thermohydrauliques de bain de corium, tout en apportant un regard critique sur ceux-ci. De ce fait, je suis actuellement pleinement conscient de la physique des accidents graves et du contexte logiciel correspondant au poste me permettant de participer à la modélisation des phénomènes physiques et son intégration dans la plateforme de calcul.

Mes motivations pour ce poste sont multiples. D'une part, le travail en équipe avec les chercheurs du LMAG et du LEAG, les interactions avec les partenaires industriels (EDF, AREVA) et chercheurs internationaux (conférences, séminaires) ont occupé une place importante durant ma thèse et ont en outre confirmé mon intérêt pour le type de recherche et le cadre de travail proposés au CEA.

Je pense pouvoir avoir un impact positif sur l'ensemble du processus de modélisation, de développement et de la validation de la plateforme PROCOR. Du fait de la diversité d'une part des phénomènes physiques (multiphysiques et multi-échelles) et d'autre part de leur modélisation inhérents au contexte des accidents graves, le développement de la plateforme et en particulier l'étude du corium en cuve est un challenge auquel j'aimerais participer. L'analyse et la validation des résultats viendront valider la qualité de la modélisation. Leur présentation permettra de valider l'expertise acquise et de mettre en valeur leur impact industriel et scientifique.

Restant à votre disposition pour toute information complémentaire, je suis disponible pour vous rencontrer lors d’un entretien à votre convenance.

Veuillez agréer, Madame, Monsieur, l’expression de mes sincères salutations.

\begin{flushright}
Louis Viot
\end{flushright}
\end{document}