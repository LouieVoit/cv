\documentclass[11pt]{article}

\usepackage[utf8]{inputenc}
\usepackage[T1]{fontenc}
\usepackage[frenchb]{babel}                   
\usepackage{hyperref}
\usepackage{geometry}
\usepackage{xcolor}
\usepackage{newpxtext,newpxmath}


\hypersetup{
  	pdfauthor = {Louis Viot},
	pdftitle = {Louis Viot : Lettre de motivation},
  	pdfsubject = {Lettre de motivation},
         colorlinks=true,
	allcolors=green!50!black,     
        urlcolor=red!50!black           
}

\geometry{
	body={6.5in, 9.0in},
	left=1.0in,
	top=1.0in
}
\setlength{\parindent}{0cm}
\setlength{\parskip}{0.5cm}

\pagestyle{empty}

\begin{document}
\begin{flushleft}
	\textbf{Louis Viot}\\
	30 Avenue Henri Pontier \\
	Batîment C2 \\
	Aix-en-Provence, 13100 France\\
	Téléphone: (+33)628685904 \\
	E-mail: \href{mailto:louis.viot@alumni.enseeiht.fr}{louis.viot@alumni.enseeiht.fr} \\
\end{flushleft}
\vspace{2 cm}
\begin{flushright}
	CEA Cadarache,\\
	13115 Saint-Paul-lez-Durance\\
\end{flushright}
\begin{flushright}
Aix-en-Provence, le \today{}.
\end{flushright}
Madame, Monsieur,

Souhaitant poursuivre ma carrière de chercheur au CEA, je me permets de vous proposer ma candidature au poste intitulé "Ingénieur développement et validation de code de calcul accidents graves" à pourvoir au sein du LMAG, laboratoire dans lequel je fini actuellement ma thèse de doctorat.

Ayant obtenu un diplôme d'ingénieur de l'\texttt{ENSEEIHT} en spécialité informatique et mathématiques appliquées, j'ai suivi de nombreux cours de développement logiciel et de numérique. Je possède donc cette double formation me permettant d'une part de résoudre des problèmes numériques liés à la modélisation des accidents graves dans la plateforme de calcul et d'autre part de participer activement au développement de son architecture logiciel. Étant de plus d'une nature curieuse et intéressé par la physique en général, je suis capable de comprendre et de modéliser les problèmes physiques résolus par la plateforme. 

Durant ces trois ans de travail, j'ai d'une part créé et intégré une nouvelle architecture logicielle au sein de la plateforme \texttt{PROCOR} permettant le couplage en temps des différents modèles physiques intervenant lors d'un accident grave et d'autre part étudié les schémas numériques permettant la résolution de ce couplage. Au travers de mon doctorat, j'ai pu mettre en pratique et étayer mes connaissances en informatique et numérique et beaucoup appris scientifiquement et en matière de relations humaines. L'expérience de thèse, où le travail en équipe, les interactions avec les partenaires industriels (EDF, AREVA) et chercheurs internationaux (conférences, séminaires) ont occupé une place importante, ont en outre confirmé mon intérêt pour le travail demandé par le poste.

Je reste à votre disposition pour vous rencontrer et vous prie de croire, Madame, Monsieur, en l'expression de mes salutations respectueuses,
\begin{flushright}
Louis Viot
\end{flushright}
\end{document}